\documentclass[paper=a4, fontsize=11pt]{scrartcl}
\usepackage[T1]{fontenc}
\usepackage{fourier}

\usepackage[english]{babel}                              % English language/hyphenation
\usepackage[protrusion=true,expansion=true]{microtype}
\usepackage{amsmath,amsfonts,amsthm} % Math packages
\usepackage[pdftex]{graphicx}
\usepackage[hyphens]{url}
\usepackage{hyperref}

% Configure referencing styles
\hypersetup{
  colorlinks=true,
  linkcolor=blue,
  filecolor=magenta,
  urlcolor=blue
}
\urlstyle{same}

%%% Custom sectioning
\usepackage{sectsty}
\allsectionsfont{\centering \normalfont\scshape}


%%% Custom headers/footers (fancyhdr package)
\usepackage{fancyhdr}
\pagestyle{fancyplain}
\fancyhead{}                      % No page header
\fancyfoot[L]{}                      % Empty
\fancyfoot[C]{}                      % Empty
\fancyfoot[R]{\thepage}                  % Pagenumbering
\renewcommand{\headrulewidth}{0pt}      % Remove header underlines
\renewcommand{\footrulewidth}{0pt}        % Remove footer underlines
\setlength{\headheight}{13.6pt}


%%% Equation and float numbering
\numberwithin{equation}{section}    % Equationnumbering: section.eq#
\numberwithin{figure}{section}      % Figurenumbering: section.fig#
\numberwithin{table}{section}        % Tablenumbering: section.tab#


%%% Maketitle metadata
\newcommand{\horrule}[1]{\rule{\linewidth}{#1}}   % Horizontal rule

\title{
  %\vspace{-1in}
  \usefont{OT1}{bch}{b}{n}
  \normalfont \normalsize \includegraphics[width=10em]{pc_h_preto} \\ [25pt]
  \horrule{0.5pt} \\[0.4cm]
  \huge Technical Advice - InterSCity Platform \\
  \horrule{2pt} \\[0.5cm]
}
\author{
  \normalfont \normalsize
  Diego Araújo Martinez Camarinha\\[-3pt]    \normalsize
  Rafael Reggiani Manzo\\[-3pt]    \normalsize
  March 19, 2019
}
\date{}


%%% Begin document
\begin{document}

\maketitle

\section{Current and former developers views on the project}
  Our first step was to reach out developers who have worked or still work with the platform. In this task, we contacted successfully the following ones: Arthur Del Esposte, who was the main developer of the platform; Dylan Guedes, currently rewriting one of the platform's services; and Higor Amario de Souza, currently responsible for the project. All of them have provided valuable feedback on the platform which are listed below:

  \begin{itemize}
    \item infrastructure stability is a consensus as the main problem with the platform currently;
    \item going further on stability, logs are difficult to find and are spread among services and servers making hard to find the source of problems;
    \item another part of the platform hard to understand is the RabbitMQ messaging because all services can read and write to it and there is no documentation on how each one performs this message exchange;
    \item there is a correct order to bring up the services which, when not followed, may lead to race conditions;
    \item the platform lacks basic security features such as authentication, authorization and permissions management;
    \item the API does not follow a standard, such as JSON API (\url{https://jsonapi.org/}), and the current one lacks a documentation;
    \item it is missing a acceptance test that assures all the platform components are working and the available hardware is enough to run the platform.
  \end{itemize}

\section{Platform installation}
  The first step on the installation process is finding the instructions. One may think that they are at \url{https://gitlab.com/interscity/interscity-platform/deploy}, but the most up to date ones were found at \url{https://github.com/LSS-USP/interscity-deploy-revoada} and only because Arthur has pointed us to them. These are the scripts that install the only known instance of the platform running currently at \url{http://playground.interscity.org/}.

  After getting the correct scripts, we have found no major issues bringing up a local installation\footnote{\label{installation-mr}\url{https://gitlab.com/pragmacode/interscity-platform/merge_requests/4}}. Below are listed minor issues that we have found:

  \begin{enumerate}
    \item the hosts must run Debian Jessie but this requirement was not listed;
    \item the scripts expect the program \textit{easy\_install} to be available at hosts and this is also not documented;
    \item if you wish to use the same scripts to perform a deployment on different hosts than Revoada's ones, there are no instructions on which settings must be changed;
    \item we found deprecation warnings\footnote{\url{https://gitlab.com/pragmacode/interscity-platform/issues/8}} raised by Ansible which can cause these scripts to fail on version 2.8 of Ansible;
    \item there are errors during the installation that can safely be ignored\footnote{\url{https://gitlab.com/pragmacode/interscity-platform/issues/7}}, but they are not documented and can lead users to think the installation did not succeed instead of looking for problems on their infrastructure.
  \end{enumerate}

  Items 1, 2 and 3 have already been addressed\textsuperscript{\ref{installation-mr}}.

  Another important remark, which we will get back to at the Discussion (\ref{sec:discussion}), is the amount of services that the platform brings up. It employs a micro-service architecture with the following components: actuator-controller; data-collector; resource-adaptor; resource-cataloguer; resource-discoverer. And these have requirements of their own: Kong; MongoDB; RabbitMQ; Redis; PostgreSQL.

  \section{Discussion}
  \label{sec:discussion}

\end{document}
