\documentclass{beamer}
\usepackage[T1]{fontenc}
\usepackage{fourier}

\usepackage[english]{babel}
\usepackage{url}
\usepackage{graphbox}

\setbeamertemplate{footline}[frame number]
\setbeamertemplate{footline}{%
  \hspace{.5em}
  \includegraphics[align=c, height=3em]{pc_h_preto}%
  \hfill%
  InterSCity Platform - Technical Advice%
  \hfill%
  \usebeamercolor[fg]{page number in head/foot}%
  \usebeamerfont{page number in head/foot}%
  \insertframenumber\,/\,\inserttotalframenumber\kern1em%
  \vspace{.25em}
}


\begin{document}

  \begin{frame}
    \frametitle{Technical Advice - InterSCity Platform}

    \begin{center}
      Download the full report at:

      \vspace{1em}

      \includegraphics[width=.45\textwidth]{qrcode}

      \vspace{.5em}

      {\tiny \url{https://gitlab.com/pragmacode/interscity-platform/wikis/uploads/43c95846f55897d6a91db8185c8878fd/report.pdf}}
    \end{center}
  \end{frame}

  \begin{frame}
    \frametitle{Objectives}

    \begin{itemize}
      \item reliability;
      \item easy of install;
      \item easy of maintain.
    \end{itemize}
  \end{frame}

  \begin{frame}
    \frametitle{Developers main concerns}

    \begin{itemize}
      \item infrastructure stability and resilience
      \item logs readability
      \item RabbitMQ
      \item startup race conditions
      \item security
      \item API standardization
      \item acceptance tests
    \end{itemize}
  \end{frame}

  \begin{frame}
    \frametitle{Platform installation}

    \begin{itemize}
      \item working as expected
      \item created local installation instructions
      \item minor issues solved or documented
      \item brings up around 10 services
    \end{itemize}
  \end{frame}

  \begin{frame}
    \frametitle{Development environment}

    \begin{itemize}
      \item we were able to setup everything
      \item spread through several repositories
      \item highly dependent on Docker and shell scripts
    \end{itemize}
  \end{frame}

  \begin{frame}
    \frametitle{Tests}

    \begin{itemize}
      \item coverage above 94\%
      \item not using mocks
      \item missing for assuring correct integration between services
    \end{itemize}
  \end{frame}

  \begin{frame}
    \frametitle{Dependencies}

    \begin{itemize}
      \item hundreds of outdated packages
      \item using Kong 0.11.2 while the current is 1.0.3
      \item rest-client has been unmaintained since 2017
      \item these are not enough to compromise the platform functionalities
        \begin{itemize}
          \item security compromised
          \item maintenance and evolution harder
        \end{itemize}
    \end{itemize}
  \end{frame}

  \begin{frame}
    \frametitle{Pipeline}

    \begin{itemize}
      \item runs tests only
      \item the release process is undefined
      \item there is room for more automation for packaging and deployment
    \end{itemize}
  \end{frame}

  \begin{frame}
    \frametitle{Additional insights}

    \begin{itemize}
      \item automatic startup
      \item up time monitoring
      \item warn users that the playground isn't a production installation
      \item logs aggregation
      \item exception notification
    \end{itemize}
  \end{frame}

  \begin{frame}
    \frametitle{Discussion}
    \frametitle{Microservices}

    \begin{itemize}
      \item it is not questioned if it is the best technical solution
      \item but is it feasible with the current development effort?
        \begin{itemize}
          \item complex deployment
          \item harder to keep dependencies updated
          \item harder to develop new features
        \end{itemize}
      \item a monolith has its own downsides
    \end{itemize}
  \end{frame}
\end{document}
